\documentclass{article}
\usepackage[utf8]{inputenc}
\begin{document}
\section*{Inlämningsuppgift nummer 1}
Av Henrik Samuelsson
\subsection*{Syfte}
I denna rapport introduceras ett valt tillverkningsflödet som kommer att analyseras grundligt under kursens senare delar. Produkten som tas fram i flödet är  ett elektroniskt system för access hantering. Systemet uppbyggnads gås igenom och de olika delsteg som behövs för att gå från delkomponenter till ett fungerade system presenteras översiktligt.

Rapporten introducerar även koncept inom ämnesområdet Lean production. Lean innebär att man försöker effektivisera sin produktion för att bli mer konkurrenskraftig. En analys enligt Lean av det valda tillverkningsflödet påbörjas.

\subsection*{Sammanfattning av teorierna}
Det finns ett begrepp som kallas ``The Toyota Way'', och har kommit att bli en synonym till Lean, detta är en uppsättning principer och beteenden som ligger till grund för produktionen i företaget Toyota. Principerna tillämpas både i företagets administrativa del och i produktionsleden. Toyota sammanfattar sin filosofi, värderingar och tillverkningsideal inom två nyckelområden: Ständiga förbättringar, och respekt för människor.

The Toyota Way innefattar fjorton olika principer. I denna inlämningsuppgift diskuteras princip två till åtta.
\subsubsection*{princip 2}


\subsection*{Resultat av tillämpning}
<Här ska dina svar på inlämningsuppgifterna skrivas. Jag bedömer speciellt förmågan att analysera,
reflektera och diskutera lean-begreppen utifrån ditt valda flöde.>
\subsection*{Diskussion}
<Här får du tillfälle att reflektera över ditt eget arbete. Gick det bra att tillämpa teorierna på ditt
flöde? Vad har du lärt dig? Etc>
\end{document}
