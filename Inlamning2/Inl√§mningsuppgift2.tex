\documentclass{article}
\usepackage[utf8]{inputenc}
\begin{document}
\section*{Inlämningsuppgift Nummer 2}
Av Henrik Samuelsson
\subsection*{Syfte}
Syftet med denna inlämningsuppgift är att göra en värdeflödesanalys på en tillverkningsprocess. Teori om bakgrunden till värdeflödesanalyser redovisas först och sedan tillämpas teorierna på ett valt flöde.

Slutresultatet redovisas dels direkt i denna rapport men kompletteras även av av en bilaga där en bild av produktionsflödet har ritas upp. I denna bild används några av de symboler som är vedertagna för att illustrera flöden av denna typ.

Syftet med värdesflödesanlyser är att identifiera och åskådligöra aktiviteter som är så kallade slöserier. Efter det att att slöserier har identifierats och redovisats följer därför även ett avsnitt innehållande åtgärder för att effektivsera flödet.

\subsection*{Sammanfattning av teorierna}
Målet med att producera en produkt är att omvandla råvaror till den slutgiltiga produkten som kunden är beredd att betala för. Arbetet som läggs ner i form av energi och kunnande för att producera produkten tillför ett mervärde och det är detta mervärde som gör att det går att driva en organisation med vinst.

Omvandlingsprocessen från råvara till färdig produkt kan i stort sett alltid delas upp i ett antal delprocesser. När man studerar dessa delprocesser så går det att dela upp dem i två olika grupper. I den första gruppen hamnar de moment som verkligen tillför ett värde till produkten. Det vill säga ett moment som verkligen är ett nödvändigt steg för att komma framåt mot slutmålet. I andra gruppen hamnar moment som inte är värdeskapande.

För att kunna vara konkurrensmässig gäller det att ha så få moment så möjligt som inte är värdeskapande. Det går aldrig helt att få bort dessa moment, men det gäller att hela tiden att arbeta för att utveckla sin tillverkningsprocess för att eliminera eller åtminstone korta ner ledtiden för dessa moment. 

\subsubsection*{Slöserier}
Icke värdeskaparande moment kallas för slöserier. Det går att dela identifiera olika typer av slöseri. På företaget Toyota har man delat valt att göra en uppdelning i följande olika typer. 
\begin{description}
\item[Överproduktion]
Huvudproblemet med överproduktion är att man riskerar att få slänga delar eftersom kunderna i framtiden inte efterfrågar just denna produkten. Sen är överprodution dåligt på flera sätt då det leder vidare till andra typer av slöserier som redovisas nedan. Det bör dock nämnas överproduktion faktiskt motverkar vissa typer av slöseri. Till exempel minskas risken för väntan genom att ha en överprodktion i form av buffertlager.
\item[Väntan]
Detta slöseri innebär att en arbetare står sysslolös. Kan orsakas av tidigare delar i flödet inte klarar av att leverera i nödvändig takt.
\item[Lager]
Att ha stora lager kostar i form av plats i fabriken och innebär att tid måste läggas på att packa ner saker som kanske måste lagras i speciell skyddande miljö. 
\item[Rörelse]
En arbetare ska kunna arbeta så smidigt så möjligt och röra sig så kort sträcka så möjligt för när hen utför sin arbetsuppgift.
\item[Omarbete]
Att behöva göra om ett moment för att det blev fel första gången. Ett annat exempel kan vara att behöva ta ut och återanvända komponenter, för att en redan monterad överproducerad produkt inte längre efterfrågas av kunderna.
\item[Överarbete]
Att tillföra funktioner till en produkt som inte eftefrågas av kunden och därmed inte tillför egentligt värde.
\item[Transporter]
Transporter externt, och internt på fabriken, tillför inget värde för kunden och ska därför minimeras.
\item[Outnyttjad kreativitet]
Man måste ta sig tid att lyssna och ta till vara på nya förslag från medarbetarna för förbättringar. Kan gälla både mindre effektiviseringar i tillverkningsprocesser eller mer genomgripande ändringar i form av produktdesign som innebär mindre materialåtgång eller snabbare flöden.  
\end{description}

\subsubsection*{Värdeflödesanalys}
Genom att rita upp en bild av en process enligt ett visst system så får man en \emph{värdeflödesanalys}. Syftet med kartläggningen är att upptäcka slöserier som kan elimineras.

En viktig komponent i värdesflödesanalysen är en tidslinje som åskådliggör tiden som går åt för värdeskappande moment relativt icke-värdeskapande moment.

Efter att ha kartlagt det flöde man har idag så brukar man även rita upp ett framtida önskat värdeflöde som vill åstadkomma genom att genomföra förändringar gällande hur man arbetar i sin process.

\subsection*{Resultat av tillämpning}
En värdesflödesanalys har gjorts av i form av en bild som bifogas som en bilaga. I denna sektion redovisas först noteringar om det studerade flödet som med fördel kan läsas parallellt medan man studerar bilden av värdeflödet.

Efter sektionerna som diskuterar underlaget för värdeflödesanlysen följer diskusioner om de olika typerna av slöserier som finns i flödet. Till sist följer åtgärder för hur man skulle kunna eliminera eller reducera ett urval av slöserierna.

\subsubsection*{Nodenheter}
Systemet som tillverkas är tvådelat där den ena delen består av en uppsättning noder. Delar till noderna anländer från ett antal olika leverantörer.

En del är ett elektroniskt styrkort som förtestas och förses med ID nummer när det anländer till prodktionsanläggningen.

Noderna är motordrivna och ett moment är att sätta fast ett kugghjul på varje motor. Efter det att kugghjulet är monterat så ska varje motor lödas fast på ett styrkort.

Styrkorten behöver även förses med mjukvara. Detta kan göras i samband med förtestet men ibland så kan detta moment behövas göras om eftersom det kommit en senare version av mjukvara under den tid som enheten ligger i mellanlager.

Elektroniken det vill säga styrkort med motor monteras sedan in i ett plasthölje tillsammans med ett ett antal rörliga delar.

Nodenheten flyttas sedan till ett mellanlager.
  
\subsubsection*{Centralenhet}
Ett komplett system består av ett antal noder kopplad till en centralenhet. Denna enhet består av ett antal olika delar som levereras av olika leverantörer. 

Centralenheten innefattar två olika elektroniskt styrkort som förtestas och förses med ID nummer när det anländer till prodktionsanläggningen. 

Varje enhet ges efter ankomsttestet en identitet i form a serienummer som programmeras in i enheten.

Centralenheten ska sedan förses med mjukvara.

Elektroniken monteras sedan in ett plasthölje.

Centralenheten flyttas sedan till ett mellanlager.

\subsubsection*{Kundorder}
När det kommer in en kundorder plockas ett antal noder fram från lagret tillsammans med en centralenhet.Dessa kopplas ihop med sladdar.

Centralenheten konfigureras upp enligt kundens önskemål och systemet testkörs sedan. 

Systemet kopplas sedan delvis isär igen och skickas vidare till en annan anläggning för slutmontering.


\subsubsection*{Ändpunkter}
Startpunkten i flödet är en uppsättning komponenter som ligger lagrade lokalt på företaget. Några av komponenterna är egenutvecklade specialanpassade specifikt för produkten, exempelvis plasthöljen och kretskort. Andra komponenter är mer av standard karaktär, såsom skruv och kablar, och kan köpas in från flera olika leverantörer.

Slutpunkten i flödet är ett monterat och konfigurerat system för förvaring som behöver vara låsbart och ofta med dynamisk accesshantering.

\subsubsection*{Organisationen} 
Huvuduppgiften för organisationen är att sköta utveckling av elektronik mjukvara och it-stöd i form av webblösningar. Antalet anställda är drygt ett dussin, varav merparten är utvecklare. Montering och konfigurering sker i nuläget ofta lokalt i organisationen men ska på sikt, när produkten är mogen, flyttas ut på golvet i fabrik.

\subsubsection*{Tolkning av principerna i valt flöde}
I denna sektion görs en återkoppling av principerna som listades ovan genom att att försöka applicera dem på det valda flödet. 

Då inte alla delar tillverkas på samma geografiska plats blir det tyvärr svårt att få till ett optimalt enstycksflöde för att så snabbt så möjligt få upp problem till ytan. Plasthöljen och komponenter formsprutas av en underleverantör medan en annan leverantör tillverkar kretskort. Till sist monteras allt i plåtsystem som tillverkas av en tredje part. Alla dessa tre moment kräver var för sig en omfattande maskinpark och expertis som i nuläget inte är rimlig att samla på en plats. På länge sikt kan man tänka sig att planera för att integrera dessa moment på en plats, för att få ett snabbare flöde. Men detta är ett omfattande projekt som inte avhandlas närmare i denna rapport. 
 
Även om det inte går att få till ett enstycksflöde som omfattar hela produktionen i närtid så går det ända att göra en del nu direkt genom att optimera delar av flödet. Det skulle gå att få till betydande förbättringar gällande princip 2 -- att bygga ett kontinuerligt processflöde som snabbt för problemen till ytan. Momenten ankomsttest, serienummerprogrammering, grundmjukvaruprogrammering, montering i hölje, grundtest (nytt test som inte görs idag) lämpar sig mycket väl för att göras i ett enstycksflöde. Förutsatt att produktionsmiljön anpassas i form av en U-formad produktionscell där all utrustning som behövs för dessa moment samlas.

När det gäller princip 3 -- att undvika överproduktion genom att använda dragande system, så används denna princip till viss del. Det går dock att göra förbättringar gällande att styra upp rutiner för vem som ansvarar för att nya delar beställs i tid. Det system som ska användas för att signalera nya beställningar behöver även konfigureras och uppgraderas.

Om man tittar på princip 4 -- jämna ut arbetsbelastningen och princip 6 -- standardisering är grunden för ständiga förbättringar och medarbetarskap, så finns det idag rutiner som ska följas för beställningar som innefattar hur kunder ska beställa och vilket resulterande datum dom beräknas kunna få sitt system levererat. Problemet här är att disciplinen när det gäller att följa dessa rutiner är dålig och det skapar stress och kvalitetsproblem när beställningar trycks igenom produktionssystemet på icke standardiserade sätt.

När det gäller princip 5 så är organisationen så pass liten i nuläget att om ett problem uppstår så stoppar det hela flödet. Detta på grund av att det är bara ett fåtal personer inblandande i det givna flödet. Sen finns det väl alltid ändå en risk att man tycker att det där är någon annans problem.

Princip 7 innefattar att ha ordning och reda i produktionen det finns här mycket att göra då utvecklingsavdelning och produktion är dåligt separerade i organisationen vilket ger oordning.

Princip 8 är problematisk att tillämpa då hela affärsidén går ut på att försöka använda ny teknik för att få konkurrensfördelar gentemot etablerade företag och metodiker. Kan nog vara så att man måste tänja lite på denna principen i detta specifika fall. Det är såklart ett vågspel men satsar man ingenting så kan man ju inte heller vinna något.

\subsection*{Diskussion}
Det är för tidigt i kursen för att kunna göra någon egentlig tillämpning av teorierna på det valda flödet. Planen framöver är först att läsa hela kursboken. Efter det så kommer det att bli fråga om att hämta in mer kunskap om flödet genom att studera arbetet på organisationen och köra intervjuer med de medarbetare som arbetar med de delar av flödet som jag personligen är minst inblandad i.
\end{document}
